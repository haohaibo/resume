% !TEX TS-program = xelatex
% !TEX encoding = UTF-8 Unicode
% !Mode:: "TeX:UTF-8"

\documentclass{resume}
\usepackage{zh_CN-Adobefonts_external} % Simplified Chinese Support using external fonts (./fonts/zh_CN-Adobe/)
%\usepackage{zh_CN-Adobefonts_internal} % Simplified Chinese Support using system fonts
\usepackage{linespacing_fix} % disable extra space before next section
\usepackage{cite}
\usepackage[colorlinks,linkcolor=blue]{hyperref}

\begin{document}
\pagenumbering{gobble} % suppress displaying page number




%\vspace{1ex}
\name{郝海波 }
%\centerline{求职意向: SDK开发工程师}
%\centerline{求职意向: GPU优化工程师}
%\centerline{求职意向: 后台开发工程师}
%\centerline{求职意向: 机器学习算法工程师}
%\centerline{求职意向: 智能平台研发工程师}
%\centerline{求职意向: 研发工程师}
%\centerline{求职意向: 云计算产品研发工程师}
%\centerline{求职意向: 创新异构计算系统软件工程师}
%\centerline{求职意向: Geospatial Big Data Infrastructure Engineer}
%\centerline{求职意向: 软件研发工程师|并行计算}
%\centerline{求职意向: 测试开发工程师}
%\centerline{求职意向: Software Developer}
%\centerline{求职意向: 算法工程师}
%\centerline{求职意向: 软件研发工程师}
%\centerline{求职意向: 服务端研发工程师}
%\centerline{求职意向: AI业务平台后端研发工程师}
%\centerline{求职意向: 后端研发工程师}
\centerline{求职意向: C++工程师}
% {E-mail}{mobilephone}{homepage}
% be careful of _ in emaill address
%\vspace{1ex}
\vspace{1ex}
%\contactInfo{haohaibo031113@163.com}{(+86) 155-1003-7852}{ \url{https://github.com/haohaibo/} }
\contactInfo{haohaibo031113@163.com}{155-1003-7852}{ \url{https://github.com/haohaibo/} }
%\contactInfo{haohaibo031113@163.com}{(+86) 155-1003-7852}{}


% {E-mail}{mobilephone}
% keep the last empty braces!
%\contactInfo{xxx@yuanbin.me}{(+86) 131-221-87xxx}{}

\vspace{-1ex}
%\vspace{1.5ex}
 
\section{\faGraduationCap\  教育背景}
\datedsubsection{\textbf{中国科学院计算技术研究所}~~ \ 硕士, 计算机系统结构}{2015 -- 2018}
\vspace{1ex}

\datedsubsection{\textbf{西安电子科技大学}~~ \ 学士, 计算机科学与技术}{2011 -- 2015}
%\vspace{1ex}

%\vspace{-1ex}

%\vspace{1ex}
%\section{\faUsers\ 项目经历}
%\datedsubsection{\textbf{Deep Learning benchmark optimization}}{2017.03 -- 现在}
%%\role{Golang, Linux}{个人项目,和富帅糕合作开发}
%\begin{onehalfspacing}
%    基于Baidu Research的开源项目DeepBench,测试深度学习框架(TensorFlow,Caffe)后端库(cuDNN,MKL)在不同
%硬件平台(GPU,MIC)上的性能。并在此基础上做卷积神经网络性能优化。
%\begin{itemize}
%  \item 在GPU和MIC端测试GEMM,Convolution, RNN的浮点峰值。分析出不同平台的性能瓶颈。为后端库开发人员提供有用的建议。
%\vspace{0.5ex}
%  \item 在我们已有的GEMM,Convolution汇编优化基础上,实现一个开源的深度学习后端库。
%\end{itemize}
%\end{onehalfspacing}

\vspace{1ex}
\section{\faUsers\ 项目经历}
\datedsubsection{\textbf{AMD GPU矩阵乘和卷积算法的优化}}{2017.04 -- 现在}
%\role{Golang, Linux}{个人项目,和富帅糕合作开发}
\begin{onehalfspacing}
     %阅读并分析AMD平台rocBLAS和深度学习后端库MIOpen。
    基于AMD CPU\&GPU异构编程平台ROCm,阅读并分析rocBLAS和MIOpen。测试MIOpen和cuDNN,Caffe(native)的性能对比。优化MIOpen,为MIOpen增加fp16的实现(进行中)。
    %基于Baidu Research的开源项目DeepBench,测试深度学习框架(TensorFlow,Caffe)后端库(cuDNN,MKL)在不同
%硬件平台(GPU,MIC)上的性能。并在此基础上做卷积神经网络性能优化。
\begin{itemize}
  %\item 在GPU和MIC端测试GEMM,Convolution, RNN的浮点峰值。分析出不同平台的性能瓶颈。为后端库开发人员提供有用的建议。
    \item 已有基础:阅读并吸收rocBLAS,rocBLAS \& MIOpenGEMM contributor。
\vspace{0.5ex}
  \item MIOpen矩阵乘和卷积kernel优化,hipCaffe代码移植(进行中)。
\end{itemize}
\end{onehalfspacing}

\datedsubsection{\textbf{TGMM细胞检测与追踪算法并行优化} }{2016.10 -- 2016.12}
%\role{实习}{经理: 高富帅}
%xxx后端开发
\begin{onehalfspacing}
   采用TGMM(Tracking with Gaussian Mixture Model)算法检测和追踪荧光显微图像中的细胞。
\begin{itemize}
   % \item 采用TGMM(Tracking with Gaussian Mixture Model)算法检测和追踪荧光显微图像中的细胞。
%\vspace{0.5ex}
    \item 在GPU端实现Median Filter(获得5倍加速),KNN和Gaussian Mixture Model计算的并行加速。
\vspace{0.5ex}
  %\item OpenMP优化,获得3.4倍的加速,用C++11 std::thread优化部分串行代码。
  \item 用C++11 std::thread优化部分串行代码(获得3.4倍加速)。
\end{itemize}
\end{onehalfspacing}

%\vspace{-1.5ex}
%\vspace{2ex}

\datedsubsection{\textbf{CATMAID-5d图像标注工具二次开发}}{2016.06 -- 2016.08}
%\role{Golang, Linux}{个人项目,和富帅糕合作开发}
\begin{onehalfspacing}
CATMAID(Collaborative Annotation Toolkit for Massive Amounts of Image Data) 是一个高效
的 web 协同标注工具。通过修改CATMAID(3d-x,y,z)源码(40k+ python,230k+ js代码)来满足标注5d(x,y,z,c,t)图像的需求。
\begin{itemize}
  \item 找出并修改了CATMAID源码的bug,解决了CATMAID从django低版本向高版本迁移的错误。
\vspace{0.5ex}
  \item 在HHMI Janelia Research Campus开源项目CATMAID的基础上做二次开发,实现5d图像多人同时在线标注的功能。
\end{itemize}
\end{onehalfspacing}

%\vspace{-1.5ex}

%\datedsubsection{\textbf{SIGHAN-2015 Chinese Spelling Check Task}}{2015.03 -- 2015.05}
%\role{\LaTeX, Python}{个人项目}
%\begin{onehalfspacing}
%该评测任务是对繁体中文进行拼写检查,给出正确的拼写结果。
%\begin{itemize}
 % \item 实现候选生成排序模型:根据同音、近音、形近字,为句中每一个繁体字生成候选, 并打分排序。
 % \item 实现两轮候选重排序模型:采用简单特征进行预排序,在第一次排序的基础上结合复杂特征进行第二次排序,以此提高排序效率。最终取得第一名,F值超出第二名18\%。
%\end{itemize}
%\end{onehalfspacing}

%\vspace{-1.5ex}

%\vspace{2ex}
%\datedsubsection{\textbf{LC-3模拟器 ISA}}{2014.07 -- 2014.08}
%%\role{\LaTeX, Python}{个人项目}
%\begin{onehalfspacing}
%LC-3模拟器汇编层次上走迷宫算法的递归实现,中断调用子程序的编写。
%\begin{itemize}
% \item 掌握LC-3 ISA。
%\vspace{0.5ex}
% \item 熟悉从基本门电路到有限状态机、ALU、控制器系统的构造LC-3模拟器的过程。
%\end{itemize}
%\end{onehalfspacing}


% Reference Test
%\datedsubsection{\textbf{Paper Title\cite{zaharia2012resilient}}}{May. 2015}
%An xxx optimized for xxx\cite{verma2015large}
%\begin{itemize}
%  \item main contribution
%\end{itemize}

\vspace{-1ex}
\vspace{1.5ex}
\section{\faSitemap\ 实习经历}
\datedsubsection{\textbf{LogInsight公司 ~日志压缩/~算法工程师}}{2016.03 -- 2016.05}
\vspace{-0.5ex}
%\role{\LaTeX, Python}{个人项目}
\begin{onehalfspacing}
%优雅的 \LaTeX\ 简历模板, https://github.com/billryan/resume
\begin{itemize}
  \item 完成论文Fast and efficient log file compression日志压缩算法的开发,相比gzip平均可以提升17\%\textasciitilde18\%的压缩比。
 % \item 基于Kafka及MySQL完成专车调度日志系统重构,解耦子模块间相互依赖,提高系统可用性。
\end{itemize}
\end{onehalfspacing}

\vspace{-1.5ex}

%\datedsubsection{\textbf{百度~网页搜索部~/~研发工程师}}{2013.12 -- 2014.05}
%\vspace{-0.5ex}
%\role{\LaTeX, Python}{个人项目}
%\begin{onehalfspacing}
%\begin{itemize}
%  \item  完成 livewdbbroom 工具多线程开发,通过测试并上线,效率提升 2 倍以上。
%  \item 完成离线时效性日志体系建设及离线时效性问题追查平台建设,并上线。
%  \item 通过 Code Master(C++) 考试,获得 Good Coder 认证。
%\end{itemize}
%\end{onehalfspacing}

\vspace{-1ex}
\vspace{1.5ex}

\section{\faTrophy\ 奖项}
\datedline{\textit{西安电子科技大学 “华为杯” 程序设计二等奖}}{2014.05}
\vspace{0.5ex}
\datedline{\textit{西安电子科技大学校二等奖学金}}{2012.08}
\vspace{0.5ex}
\datedline{\textit{西安电子科技大学数学建模校赛二等奖}}{2012.06}
%\datedline{\textit{银牌 (\nth{17}/200)}~ACM-ICPC 国际大学生程序设计竞赛亚洲区域赛北京赛区}{2015.11}
%\datedline{\textit{四强 (\nth{4}/2293)}~阿里巴巴大数据竞赛“新浪微博互动预测大赛”}{2015.12}
%\datedline{\textit{一等奖}~华北五省及港澳台大学生计算机应用大赛}{2013.11}
%\datedline{\textit{一等奖}~全国软件专业人才设计与创业大赛北京赛区}{2012.04}

\vspace{-1ex}
\vspace{1.5ex}

\section{\faCogs\ 个人能力}
% increase linespacing [parsep=0.5ex]
\begin{itemize}[parsep=0.5ex]
  %\item 了解 C++、CUDA 、Python。
  \item 语言C > C++ > CUDA = Python。
  \item 熟悉基本数据结构和算法,有良好的编程风格。
  \item 了解NVIDIA GPU架构、熟悉CUDA。
  \item 了解MPI, MapReduce。
  \item 英语: CET-6 
%  \item 熟悉数据挖掘、机器学习领域基本算法。
\end{itemize}

%\section{\faInfo\ 其他}
%% increase linespacing [parsep=0.5ex]
%\begin{itemize}[parsep=0.5ex]
%  \item 技术博客: http://blog.yours.me
%  \item GitHub: \url{https://github.com/haohaibo}
%  \item 英语: CET-4 489 CET-6 426 
%\end{itemize}

%% Reference
%\newpage
%\bibliographystyle{IEEETran}
%\bibliography{mycite}
\end{document}
