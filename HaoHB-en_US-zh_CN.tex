% !TEX TS-program = xelatex
% !TEX encoding = UTF-8 Unicode
% !Mode:: "TeX:UTF-8"

\documentclass{resume}
\usepackage{zh_CN-Adobefonts_external} % Simplified Chinese Support using external fonts (./fonts/zh_CN-Adobe/)
%\usepackage{zh_CN-Adobefonts_internal} % Simplified Chinese Support using system fonts
\usepackage{linespacing_fix} % disable extra space before next section
\usepackage{cite}
\usepackage[colorlinks,linkcolor=blue]{hyperref}

\begin{document}
\pagenumbering{gobble} % suppress displaying page number


%\name{Jianpeng Hou}
\name{Haibo Hao}
%\centerline{Machine Learning; Software Development}
\centerline{Software Development}
\vspace{2ex}
% {E-mail}{mobilephone}{homepage}
% be careful of _ in emaill address
% {E-mail}{mobilephone}
% keep the last empty braces!
%\contactInfo{houjp1992@gmail.com}{(+86) 152-0000-0000}{ \url{https://github.com/houjp/}}
\contactInfo{haohaibo031113@163.com}{155-1003-7852}{ \url{https://github.com/haohaibo/}}

%\section{\faCogs\ Skills}
 
\section{\faGraduationCap\ Education}
%\datedsubsection{\textbf{Institute of Computer Technology, Chinese Academy of Sciences}}{Sep. 2014 -- Jul. 2017}
\datedsubsection{\textbf{Institute of Computer Technology, Chinese Academy of Sciences}}{Sep. 2015 -- Jul. 2018}
%\textit{Master student} in Computer Software \& Theory, Rank: top 5\%
\textit{Master student} in Computer Architecture
%\datedsubsection{\textbf{University of Science \& Technology Beijing}}{Sep. 2010 -- Jul. 2014}
\datedsubsection{\textbf{Xidian University}}{Sep. 2011 -- Jul. 2015}
%\textit{Bachelor student} in Computer Science \& Technology, Rank: \nth{4}/124
\textit{Bachelor student} in Computer Science \& Technology

\section{\faUsers\ Project Experience}
%\datedsubsection{\textbf{Big Data Analysis Platform (\href{http://159.226.40.104:18080}{http://159.226.40.104:18080})}}{Oct. 2015 -- Mar. 2016}
%AMD GPU Matrix Multiplication and Convolution Algorithm Optimization 2017.04 - Now
%Based on AMD CPU & GPU heterogeneous programming platform ROCm, read and analyze rocBLAS and MIOpen. Test MIOpen and cuDNN, Caffe (native) performance comparison. Optimize MIOpen, add fp16 implementation for MIOpen (in progress).

%• Basics: read and absorb rocBLAS, rocBLAS & MIOpenGEMM contributor.
%• MIOpen matrix multiplication and convolution kernel optimization, hipCaffe code migration (in progress)

%\datedsubsection{\textbf{Matrix Multiplication and Convolution Algorithm Optimization on AMD GPU}}{Apr. 2017 -- Now}
\datedsubsection{\textbf{GEMM and Convolution Algorithm Optimization on AMD GPU}}{Apr. 2017 -- Now}
%\role{Summer Intern}{Manager: xxx}
I read and analyzed the source code of rocBLAS and MIOpen on AMD GPU heterogeneous computing platforms. I tested the peak performance comparison between MIOpen and cuDNN, Caffe(native). I am optimizing the MIOpen library and adding fp16 implementations of GEMM \& Conv to MIOpen.
%Based on AMD CPU \& GPU heterogeneous programming platform ROCm, I read and analyze rocBLAS and MIOpen source code. Test MIOpen and cuDNN, Caffe (native) performance comparison. Optimize MIOpen, add fp16 implementation for MIOpen (in progress).
\begin{itemize}
  %\item  Developed distributed algorithms(CART/GBDT/GBRT/RF) on Spark.
    %\item  Basics: read and absorb rocBLAS, rocBLAS \& MIOpenGEMM contributor.
    \item  rocBLAS and MIOpen GitHub contributor.
  %\item Finished data mining components(Feature-Indexing/Feature-Merging/Feature-Normalization/Scoring).
    \item MIOpen matrix multiplication(GEMM) and convolution kernel optimization, hipCaffe code migration (in progress).
%  \item Optimized xxx 5\%
\end{itemize}

%\datedsubsection{\textbf{China Telecom Big Data Application Contest (\nth{1}/1112; Team Leader)}}{Dec. 2015 -- Mar. 2016}
\datedsubsection{\textbf{TGMM Cell Detection and Tracking Algorithm Parallel Optimization}}{Dec. 2016 -- Mar. 2017}
%\role{C, Python, Django, Linux}{Individual Projects, collaborated with xxx}
%The goal of this contest is to predict views of users with ten sites succeeding, according to four hundred million user-behavior historical records(25.38G).
Detecting and tracking cells in the fluorescence microscopic images by TGMM (Tracking with Gaussian Mixture Model) algorithm.
%Brief introduction: xxx
\begin{itemize}
  %\item Proposed and implemented a multi-target regression algorithm on Spark. Optimized F1-score 0.8\%.
    \item Parallel acceleration on the GPU with Median Filter (5x acceleration), KNN and Gaussian Mixture Model calculations.
  %\item Designed and developed a probability ranking model for user classification. Optimized F1-score  0.6\%.
    \item Optimize some serial code with C++11 std::thread (3.4x acceleration).
%  \item xxx
\end{itemize}

%\datedsubsection{\textbf{SIGHAN-2015 Chinese Spelling Check Task (\nth{1} Place)}}{Mar. 2015 -- May. 2015}
\datedsubsection{\textbf{CATMAID-5d Image Tracing Tool Secondary Development}}{Jun. 2016 -- Sep. 2016}
%\role{\LaTeX, Maintainer}{Individual Projects}
%The goal of this task is to detect and correct spelling errors on Chinese essays.
CATMAID (Collaborative Annotation Toolkit for Massive Amounts of Image Data) is an efficient web collaboration annotation tool. I modified CATMAID source code(40k+ python, 230k+ js code) to meet to requirements of 5d image annotation.%The requirements of the 5d (x, y, z, c, t) image are met by modifying the CATMAID (3d-x, y, z) source (40k+ python, 230k+ js code).
\begin{itemize}
  %\item Handled this task with a unified framework which consisted of candidate generating, two stage candidates re-ranking and global decision making.
  %\item Identify and modify the bugs of the CATMAID source code to resolve the bug that CATMAID migrated from the django low version to the high version.
  \item I fixed the bug when migrating CATMAID from the django lower version to high version.
  %\item Finished candidate generating model and two stage candidates re-ranking model. Optimized F1-score 18\%.
  \item On the basis of HHMI Janelia Research Campus open source project CATMAID , I do the second development, implemente the function of multi-person annotation simultaneously.
\end{itemize}

\section{\faSitemap\ Intern Experience}

\datedsubsection{\textbf{Machine daily log file compression at LogInsight}}{Mar. 2016 -- May. 2016}
%\role{\LaTeX, Maintainer}{Individual Projects}
%The goal of this task was to detect and correct spelling errors on Chinese essays.
\begin{itemize}
  %\item Developed taxi dispatching system which used to balance the supply and demand between urban areas.
    \item Implemented the algorithm described in papter \emph{Fast and efficient log file compression}. Improving compression ratio by 17\%\textasciitilde18\% compared to gzip.
  %\item Reconstructed the log system of dispatching based on Kafka and MySQL.
\end{itemize}

%\datedsubsection{\textbf{Baidu Online Network Technology(Beijing) Co.,Ltd}}{Dec. 2013 -- May. 2014}
%\role{\LaTeX, Maintainer}{Individual Projects}
%The goal of this task was to detect and correct spelling errors on Chinese essays.
%\begin{itemize}
%  \item Completed the multi-threaded development of LiveWDBBroom, improved the efficiency by two times.
%  \item Completed the development of the Problems-Tracing Platform which was put into service.
%  \item Passed the examination of Code Master(C++), certified as Good Coder.
%\end{itemize}


% Reference Test
%\datedsubsection{\textbf{Paper Title\cite{zaharia2012resilient}}}{May. 2015}
%An xxx optimized for xxx\cite{verma2015large}
%\begin{itemize}
%  \item main contribution
%\end{itemize}


%\section{\faTrophy\ Academic Competitions}
%\datedline{\textit{\nth{1} Place}~~Awarded in China Telecom Big Data Application Contest}{Mar. 2016}
%\datedline{\textit{\nth{1} Place}~~Awarded in SIGHAN-2015 Chinese Spelling Check Task}{Jun. 2015}
%\datedline{\textit{\nth{1} Place}~~Awarded in China College Students Computer Games Competition}{Nov. 2013}
%\datedline{\textit{\nth{1} Prize}~~Awarded in China College Students Computer Application Contest}{Nov. 2013}
%\datedline{\textit{\nth{1} Prize}~~Awarded in "LanQiao Cup" Software Development Contest(Beijing Division)}{Apr. 2012}
%\datedline{\textit{Silver Medal~(\nth{17} / 200)}~~Awarded in ACM-ICPC Asia Beijing Regional Contest}{Nov. 2015}

%\section{\faTrophy\ Academic Competitions}
\section{\faTrophy\ Awards}
%\datedline{\textit{\nth{2} Place}~~Awarded in "Huawei Cup" programming Contest}{Mar. 2014}
\datedline{Second price in the "Huawei Cup" programming contest}{May. 2014}
%\datedline{\textit{\nth{2} Place}~~Awarded in Xidian University scholarship}{Sep. 2012}
\datedline{Xidian University second-class scholarships}{Sep. 2012}
%\datedline{\textit{\nth{2} Place}~~Awarded in Xidian University Mathematical Competition}{Nov. 2013}
\datedline{Second prize in Xidian University Mathematical modeling competition}{Jun. 2012}
%\datedline{\textit{\nth{1} Prize}~~Awarded in China College Students Computer Application Contest}{Nov. 2013}
%\datedline{\textit{\nth{1} Prize}~~Awarded in "LanQiao Cup" Software Development Contest(Beijing Division)}{Apr. 2012}
%\datedline{\textit{Silver Medal~(\nth{17} / 200)}~~Awarded in ACM-ICPC Asia Beijing Regional Contest}{Nov. 2015}

\section{\faCogs\ Skills}
\begin{itemize}[parsep=0.5ex]
  %\item Skilled in C++, Scala, Shell. Familiar with data structures and algorithms and had good programming style.
  \item Skilled in C, C++, CUDA, Python. Familiar with data structures and algorithms and had good programming style.
  %\item Experienced in development of distributed machine learning algorithms.
  \item Understanding the NVIDIA GPU architecture.
  %\item Strong theoretic knowledge on data mining and machine learning.
  \item English : CET6.
\end{itemize}


\end{document}
